% Options for packages loaded elsewhere
\PassOptionsToPackage{unicode}{hyperref}
\PassOptionsToPackage{hyphens}{url}
%
\documentclass[
]{article}
\usepackage{amsmath,amssymb}
\usepackage{lmodern}
\usepackage{iftex}
\ifPDFTeX
  \usepackage[T1]{fontenc}
  \usepackage[utf8]{inputenc}
  \usepackage{textcomp} % provide euro and other symbols
\else % if luatex or xetex
  \usepackage{unicode-math}
  \defaultfontfeatures{Scale=MatchLowercase}
  \defaultfontfeatures[\rmfamily]{Ligatures=TeX,Scale=1}
\fi
% Use upquote if available, for straight quotes in verbatim environments
\IfFileExists{upquote.sty}{\usepackage{upquote}}{}
\IfFileExists{microtype.sty}{% use microtype if available
  \usepackage[]{microtype}
  \UseMicrotypeSet[protrusion]{basicmath} % disable protrusion for tt fonts
}{}
\makeatletter
\@ifundefined{KOMAClassName}{% if non-KOMA class
  \IfFileExists{parskip.sty}{%
    \usepackage{parskip}
  }{% else
    \setlength{\parindent}{0pt}
    \setlength{\parskip}{6pt plus 2pt minus 1pt}}
}{% if KOMA class
  \KOMAoptions{parskip=half}}
\makeatother
\usepackage{xcolor}
\usepackage[margin=1in]{geometry}
\usepackage{graphicx}
\makeatletter
\def\maxwidth{\ifdim\Gin@nat@width>\linewidth\linewidth\else\Gin@nat@width\fi}
\def\maxheight{\ifdim\Gin@nat@height>\textheight\textheight\else\Gin@nat@height\fi}
\makeatother
% Scale images if necessary, so that they will not overflow the page
% margins by default, and it is still possible to overwrite the defaults
% using explicit options in \includegraphics[width, height, ...]{}
\setkeys{Gin}{width=\maxwidth,height=\maxheight,keepaspectratio}
% Set default figure placement to htbp
\makeatletter
\def\fps@figure{htbp}
\makeatother
\setlength{\emergencystretch}{3em} % prevent overfull lines
\providecommand{\tightlist}{%
  \setlength{\itemsep}{0pt}\setlength{\parskip}{0pt}}
\setcounter{secnumdepth}{-\maxdimen} % remove section numbering
\ifLuaTeX
  \usepackage{selnolig}  % disable illegal ligatures
\fi
\IfFileExists{bookmark.sty}{\usepackage{bookmark}}{\usepackage{hyperref}}
\IfFileExists{xurl.sty}{\usepackage{xurl}}{} % add URL line breaks if available
\urlstyle{same} % disable monospaced font for URLs
\hypersetup{
  pdftitle={To Bet or Not to Bet: Analyzing the Credibility of Fixed-odds Betting on Match Outcomes},
  pdfauthor={Tseegi Nyamdorj, Fungai Jani, Maria Tsakalakos},
  hidelinks,
  pdfcreator={LaTeX via pandoc}}

\title{To Bet or Not to Bet: Analyzing the Credibility of Fixed-odds
Betting on Match Outcomes}
\author{Tseegi Nyamdorj, Fungai Jani, Maria Tsakalakos}
\date{2023-07-24}

\begin{document}
\maketitle

{
\setcounter{tocdepth}{2}
\tableofcontents
}
\hypertarget{introduction}{%
\subsection{Introduction}\label{introduction}}

The world of soccer is constantly growing, with millions of fans all
over the globe. With the growing fanbase comes a growing betting
community. Soccer accounts for 70\% of all global sports bets placed. We
focus specifically on the Premier League, as it is the most watched
soccer league. A number of the Premier League clubs have betting
companies as sponsors on their kits, which encourages supporters to
place bets. Are the odds for each match's outcome fair? Do the betting
favorites tend to come out on top in the Premier League? Attempting to
predict Premier League match outcomes will have real world implications
on the betting market, journalism, and coaching decisions.

Previous studies have used Bayesian nets models to predict match results
in Premier League seasons, with accuracies floating around 40\%. The
majority of the models were not good at predicting the matches. These
suboptimal performances indicate the complexity associated with
forecasting football match outcomes. Gambling organizations also
prioritize match prediction probabilities to provide oddsmakers. By
using the data from these models, oddsmakers may create fair and
competitive betting settings for the general public by providing them
with educated and balanced odds. This highlights the need for robust and
innovative approaches in this area.

This paper aims to build a comprehensive predictive framework for
forecasting match results in the Premier League season of 2018-2019. To
build upon the already existing foundation of betting odds we enriched
our predictive models by adding two key components: player evaluation
metrics and the recent form of the teams based on their performances
over the last five matches. The addition of player evaluation metrics is
to give a better gauge of the strength of each team and where the
strength lies in comparison to its opposition. We aim to contribute
significant insights that can help sports analysts, coaches and
enthusiast alike in making more informed decisions and understanding the
dynamics of competitive football matches in one of the most esteemed
leagues in the world through a thorough evaluation of three different
models: Generalized Additive Models, Random Forest and Multinomial
Models.

\hypertarget{exploratory-data-analysis-data-summary}{%
\subsection{Exploratory Data Analysis \& Data
Summary}\label{exploratory-data-analysis-data-summary}}

\hypertarget{the-data}{%
\subsubsection{The Data:}\label{the-data}}

For our research, we utilized two publicly available data sets from the
Premier League 2018-2019 season. The first data set was obtained from
\url{footstats.org}, accessible under the ``Football Stats Database to
CSV and Excel'' section on the website. This CSV file contained
game-level data, with each entry representing a specific game from the
2018-2019 season. The data set comprised approximately 72 columns,
providing detailed information about various events occurring in each
match. These variables covered a wide range of aspects, including the
names of the home and away teams, the full-time result of the match,
expected goals, shots on target for the home team, as well as additional
details like the referee's name, the stadium where the match was played,
and the timestamp of the game.

The second data set we used was betting data from football-data.co.uk,
where we were able to obtain free soccer betting odds \& results, match
statistics, odds comparison, and more. Similar to the previous data set,
we specifically downloaded data pertaining to the Premier League's
2018-2019 season to ensure compatibility with our existing dataset. The
football-data.co.uk data set differed from the previous one in that it
contained betting odds from various betting sites. However, for the
purpose of our research, we focused solely on the data from Bet 365
odds, allowing us to maintain consistency throughout our analysis.

After obtaining the two data sets, we merged them by the home team, away
team, and the date of each match resulting in a data set with 136
columns and over 300 observations. Then, we dropped all the columns that
were unnecessary for our analysis. Also, we cleaned the data by renaming
column names and by fixing the format of the match date.

After processing the data, we added several additional variables:

\begin{itemize}
\tightlist
\item
  expect\_win: created a new expected win variable based on the betting
  data
\item
  expect\_point\_h: expected home points from the betting odds

  \begin{itemize}
  \tightlist
  \item
    We computed the expected home points based on
    \url{casino.betmgm.com}. It is important to note that the betting
    data split the odds into three columns for each game: home team
    odds, away team odds, and draw odds. So, our first step was to
    convert the betting odds to fractional odds. To do this, we created
    a new variable based on the home team winning odds, the away team
    winning odds, and the draw odds. The variables were computed by
    1/betting odd. Then, by adding the three variables together, we
    computed the marginal total probability for each game. Next, to
    convert this number into a percentage, you divide the fractional odd
    by the total marginal probability, resulting in a percentage of each
    game outcome that reflects the betting data. Finally, to achieve the
    variable expect\_point\_h, you use this equation
    \(percent_H \cdot 3 + percent_D\).
  \end{itemize}
\item
  Full\_time\_result\_point: taking the actual outcome of the match and
  converting it into a dummy variable. The home team loses = 0, the home
  team draws = 1, the home team wins = 3.
\item
  expect\_point\_diff\_bet: Difference between the expected for the home
  team minus the away team based on the betting data.
\item
  streak: the winning streak of each team from the last five games,
  weighted by historical data.
\end{itemize}

Although the merged datasets already grant us ample information to run
our models effectively, we have taken innovative steps to enhance our
prediction further. This entails the incorporation of the variable
consisting of a team ranking. Here, we took player ranking data from
FIFA.com for the 2018-19 season, filtered players from the Premier
League only, and then computed the position ranking within each team.
The ranking process involved several steps. First, we considered the
number of minutes played by each player in the season. Players with
higher minutes played received a higher weight since they are more
likely to have a larger impact on the team. Secondly, we grouped the
players into their respective position (forwards, midfielders,
defenders, and goalkeepers), allowing us to further rank the positions
on each team. From there, we were then able to compute an overall team
ranking, which serves as a valuable metric in our analysis.

\hypertarget{methods}{%
\subsection{Methods}\label{methods}}

Prior to applying all explanatory parameters to our models, we wanted to
first test-out the predictability of match outcomes through betting-odds
alone. As the dependent variable (match outcome) is an ordered
categorical variable, the models we could utilize were inherently
limited to main three models: Generalized Additive Models (GAMs),
Multinomial regressions, and Random forest algorithms. After assembling
the base models, we added more independent parameters to increase the
accuracy prediction of our models. In total, we used four different
explanatory variables, namely the streak difference (the last five
games' results of each team leading up to that specific match), expected
goals (XGs) for the home team computed by the betting odds, XGs
difference, and XGs difference based on defense and offense rating of
home/away teams.

Then models were trained on matches from the start of the season until
February of the 2018/19 season as it encompasses \(2/3\) of the season
and allows machine learning to best adapt the available data, which then
can be used to test out the remaining 2018/19 season matches.

\begin{itemize}
\tightlist
\item
  \textbf{Multinomial regression}: Firstly, we fit multinomial models as
  the response variable is a categorical data that requires using a
  linear combination of the observed variables to estimate the
  probability of each classified value of the dependent parameter. A
  compact version of the prediction function is:
\end{itemize}

\(f(k, i) = \beta_k + x_i\) where \(\beta_k\) is the set of regression
coefficients associated with outcome k, and \(x_i\) (row vector) is the
set of explanatory variables associated with observation \(i\).

\begin{itemize}
\item
  \textbf{Generalized Additive Models (GAMs)}: We fit GAMs with various
  combinations of explanatory variables since some parameters could
  result in the model overfitting on the training data. As mentioned
  previously, the response variable being an ordered, non-binary
  categorical data meant that we had to use multinomial GAMs with K+1
  categories for K linear predictors. The likelihood of each response
  variable can be transcribed in the following way:
  \[\frac{e^{\eta_y}}{1 + \Sigma_j e^{\eta_i}}\] where \(\eta_y\) is the
  linear predictor of \(y\) category of the dependent variable.
\item
  \textbf{Random forest algorithms}: Lastly, we utilized the Random
  Forest to generate predictions of different categories of the
  dependent variable. To quickly implement random forest, we used the
  ranger package with classification method.
\end{itemize}

\hypertarget{results}{%
\subsection{Results}\label{results}}

\hypertarget{conclusion-limitation}{%
\subsection{Conclusion (limitation)}\label{conclusion-limitation}}

\hypertarget{figures-and-tables}{%
\subsection{Figures and Tables}\label{figures-and-tables}}

\hypertarget{acknowledgments}{%
\subsection{Acknowledgments}\label{acknowledgments}}

\hypertarget{reference}{%
\subsection{Reference}\label{reference}}

\end{document}
